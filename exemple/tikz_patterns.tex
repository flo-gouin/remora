\tikzstyle{choice}=[diamond,draw,aspect=3, inner sep=0.1ex, scale=1.0]
\tikzstyle{node}=[rectangle,draw, align=center, inner sep=1.0ex, rounded corners=4pt, scale=1.0]
\tikzstyle{nodebox}=[rectangle,draw, align=center, inner sep=1.0ex, scale=1.0]
\tikzstyle{info}=[align=center, inner sep=1.0ex, scale=1.0]
\tikzstyle{io}=[info, rounded corners=5pt, fill=gray, text=white]
\tikzstyle{link}=[->,>=latex,very thick]
\tikzstyle{box}=[draw, dashed, ultra thick, gray, rounded corners=6pt]
\tikzstyle{boxName}=[align=right, inner sep=.2ex, scale=1.0,color=blue, right, rotate=-90]

\newcommand{\namedBox}[3]%content, text, color
%{\node[box, inner sep=20pt, fit=#1, label={[align=left,shift={(-4.0ex,-1.0ex)},rotate={(-90)}, #3]north east:#2}]{};}
{\node[box, inner sep=10pt, fit=#1, label={[align=right,xshift={(-0.5ex)},io,rotate={(-90)}]north east:#2}]{};}

\newcommand{\thinNamedBox}[3]%content, text, color
{\node[box, inner sep=5pt, fit=#1, label={[align=right,xshift={(-0.5ex)},io,rotate={(-90)}]north east:#2}]{};}

\newcommand{\namedIdBox}[4]%content, text, color, id
{\node[box, inner sep=10pt, fit=#1, label={[align=right,xshift={(-0.5ex)},io,rotate={(-90)}]north east:#2}] (#4) {};}

%========================================================Experimental data graphs
%\usetikzlibrary{decorations.text}
\newcommand{\sectionElt}[5]{%tin, tout, level, name, color
	\pgfmathparse{#3 +0.4}\let\z\pgfmathresult;
%	\draw[fill=#5] (#1:#3) arc (#1:#2:#3) node[midway,sloped,above]{#4} -- (#2:\z) arc (#2:#1:\z) -- cycle;
	\draw[fill=#5] (#1:#3) arc (#1:#2:#3) -- (#2:\z) arc (#2:#1:\z) -- cycle;

%	\path[decoration = {text along path, text = {#4}, text align = {align = center}, raise = -0.5ex}, decorate] (#1:#3) arc (#1:#2:#3);

%	\pgfmathparse{(#1+#2)/2}\let\x\pgfmathresult;
%	\draw (\x:\z) -- (\x:9) node{test};
%	\node[sloped, above,align=center] at (\x:#3) {Test};
%	\path [decorate,decoration={text along path, text={test}}] (#1:#3) arc (#1:#2:#3);
	
%	[postaction={decorate},decoration={text along path,text={#4},text align=center}]
}

\newcommand{\fctSection}[4]{%tin, tout, level, name, color
	\pgfmathparse{#3 +0.4}\let\z\pgfmathresult;
%	\pgfmathparse{#3 +0.3}\let\w\pgfmathresult;
%	\draw[fill=#5] (#1:#3) arc (#1:#2:#3) node[midway,sloped,above]{#4} -- (#2:\z) arc (#2:#1:\z) -- cycle;
	\draw[fill=gray!40] (#1:#3) arc (#1:#2:#3) -- (#2:\z) arc (#2:#1:\z) -- cycle;
	
%	\path [decorate,decoration={text along path, text={#4}}] (#1:\w) arc (#1:#2:\w);
	
%	\pgfmathparse{(#1+#2)/2}\let\w\pgfmathresult;
%	\draw (\w:\z) -- (\w:8) node{#4};
%	\node at(\w:8){#4};
}

%Old
\newcommand{\gpuSection}[4]{%tin, tout, level, name
%	\sectionElt{#1}{#2}{#3}{#4}{green!75};
	\pgfmathparse{#3 +0.4}\let\z\pgfmathresult;
	\draw[fill=green!75] (#1:#3) arc (#1:#2:#3) node[midway,sloped]{#4} -- (#2:\z) arc (#2:#1:\z) -- cycle;
}

\newcommand{\sectionIter}[2]%time, level
{\pgfmathparse{#2 +0.4}\let\z\pgfmathresult;\draw[dashed] (#1:#2) -- (#1:\z);}

\newcommand{\sectionLevel}[1]%level
{\pgfmathparse{#1 +0.2}\let\z\pgfmathresult;\draw[color=gray] (0:\z) arc (0:330:\z);}

%========================================================
